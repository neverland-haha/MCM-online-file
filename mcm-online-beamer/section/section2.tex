\section{养成一些好习惯}
\subsection{遇事不决先问文档再问百度}

\begin{frame}{遇事不决先问文档再问百度}
	\begin{columns}
		\column{0.4\textwidth}<1->
		\includegraphics[width=0.9\textwidth,height=0.5\textheight]{manual.png}
		\column{0.6\textwidth}<1->
		\includegraphics[width=0.9\textwidth,height=0.5\textheight]{google.png}
	\end{columns}	~\\
{
	\textcolor{red}{遇到事情,先找找相关宏包的手册,再去问百度能否解决,能用 Google 解决尽量用 Google解决。对于刚入门的\LaTeX 的人来说,《112分钟了解\LaTeX》已经能够解决大部分问题。}
}		
	
\end{frame}

\subsection{学会提问}
\begin{frame}{提问是一门艺术}
	\begin{itemize}
		\item 提问之前最好先尝试自己解决问题。
		\item 如果无法解决,请提供 MWE(Minimal Working Example)以及报错信息、日志等材料,并提供你想实现的
		效果。
		\item 学会自己去阅读报错信息,并合理利用大段注释法解决问题。
		\item 找到合适的渠道提问,例如 师兄师姐、QQ 群、小屋中的问答模块,texstackexchange 中的等等。\\[1cm]
		
		{	\centering
			\textcolor{red}{只要态度认真,提问内容清晰完整,再加上适当的礼貌,在互联网上得到帮助都不难。}
		}
	\end{itemize}
\end{frame}

\subsection{内容与格式分离}
\begin{frame}{最基本的层次与最高的奥义}
	
	
	\textcolor{red}{\bfseries 内容与格式分离},\LaTeX 模板已经将基本的设置帮你设置完成,请一定不要自己再去
	修改某些设定,建模比赛就好比一次投稿,如果你不想麻烦评委老师,也不想因为自己的强迫症影响了格式而扣
	分,就请一定不要自行修改模板的内容。修改模板是非常困难的一件事情,对于初学者来说,无法修改出想要的
	效果不说,再咨询他人想实现效果的时候可能也没法提问。	
	\begin{figure}
		\centering
		\includegraphics[width=2.5cm,height=2.5cm]{LaTeX.jpg}
	\end{figure}
	\textcolor{red}{\bfseries \LaTeX 不是 Word、WPS},所见即所得不是它的代名词,接受它,要不放弃它。
\end{frame}


%------------------------------------------------



%第四节
\section{关于学习\LaTeX 的各种途径与资源}
\subsection{资源}
	\begin{frame}{资源}
		\begin{itemize}
			\item QQ 1群:91940767
			\item QQ 3群:640633524
			\item LaTeX 工作室:\url{https://www.latexstudio.net}.
			\item ctan:\url{https://www.ctan.org}
			\item texstackexchange:\url{https://tex.stackexchange.com}
			\item LaTeX 工作室公众号	
			\begin{figure}
				\centering
				\includegraphics[scale=0.2]{gongzhonghao.jpg}
			\end{figure}
		\end{itemize}
	
	\phantom{幻影幻影幻影幻影} \textcolor{red}{合理利用资源,永远保持一颗学习的心。}
	\end{frame}