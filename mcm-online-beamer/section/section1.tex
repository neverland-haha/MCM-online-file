\section{让\LaTeX 跑起来}

%------------------------------------------------第三节第一节发行版
\subsection{发行版}
\begin{frame}{发行版}
	\LaTeX、\TeX 系统都是复杂的软件包,里面包含各种各样的排版引擎、编译脚本、格式转换工具、配置文件、
	支持工具、字体以及文档。一个 \TeX 发行版就是把所有这样的部件都集合起来,打包发布的软件。
	\begin{itemize}
		\item \TeX Live ,跨平台,有比较齐全的宏包。
		\item Mik \TeX 临时自动下载没有所需要的宏包,体积较小。
		\item Mac \TeX 专门为 Mac 定制的发行版
		\item \textcolor{red}{\sout{C\TeX}}是一款过时的发行版,早在2012年就停止维护,对于 \TeX 来说,两年一更
		是必要的,\textcolor{red}{所以请放弃使用C\TeX。}
	\end{itemize}
	关于下载,可以参考王然老师写的 Install-LaTeX.pdf,在官网或者相应的镜像网站中下载,也可以参考 LaTeX 工
	作室写的文章。\\
	\url{https://www.latexstudio.net/archives/51801.html},发行版下载完成后,请在命令行中使用 |tex -v| 检查是否安/
	装成功。
\end{frame}

%------------------------------------------------第三节第二节编辑器
\subsection{编辑器}
\begin{frame}{编辑器}
	\begin{itemize}
		\item 命令行,用命令行进行编译能够让人能够更加理解编译的链条。
		\item WinEdt,收费软件,推荐购买后使用,对初学者界面友好。
		\item TeXworks -常见的 TeX 套装都自带这款编辑器,界面比较清爽,支持代码和 pdf 查看,左右分屏显示。
		\item TeXStudio  开源免费的编辑器,界面集成度好。
		\item Vscode 轻量级的编辑器,可以在内写多种语言,但是配置稍许麻烦。
		\item Sublime 与前者一样,也是一款轻量级的编辑器,可以在内书写多种语言,但是同样配置稍许麻烦。
	\end{itemize}
	\begin{itemize}
		\item overleaf 在线编辑 \LaTeX 的网站,投稿人常用的网站,能够省去配置的烦恼,但是由于内地使用不稳定,
		需要按照自己的情况选取。
	\end{itemize}
	
	\textcolor{red}{如果发行版与编辑器是飞机和引擎的话,那么发行版就是引擎,它让飞机飞起来,而飞机只是一
		个外框,它并不起决定作用。}
\end{frame}