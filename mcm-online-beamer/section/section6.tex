%表格章节
\section{表格的使用}
\subsection{表格的基础}

%基础使用
\begin{frame}[fragile]{表格的基础使用}
	\phantom{幻影} \textcolor{blue}{表格是 \LaTeX 中的基础知识,同时也是难点之一,对于基础表格排版比较简单
	而对于过宽过长的表格则需要的一定的基本功才能解决!}
tabular  的一般格式为:

\begin{lstlisting}
\begin{tabular}[<垂直对齐>]{<列格式说明>}
<内容>  &  <内容>  &  内容  \\
...
\end{tabular}
\end{lstlisting}
列对齐格式有:
\begin{itemize}
	\item l  左对齐 
	\item c  居中
	\item r  右对齐
	\item |p{宽}| 固定宽度
	\item |  画一条竖线
	\item |@{<内容>}| 添加任意内容,但是会取消表列的距离
	\item |*{<计数>}{<列格式说明>}|,符号重复多次
\end{itemize}
\end{frame}


%表格的基础使用

\begin{frame}[fragile]{表格的基础使用}

{
\begin{LTXexample}[pos=r,scaled]
\begin{tabular}{|c|c|c|}
\hline
第一列  &  第二列  & 第三列  \\
\hline 
第一列  &  第二列  & 第三列  \\
\hline 
\end{tabular}
\end{LTXexample}
}
三线表的使用
\smallskip
{\setlength\ResultBoxSep{3mm}
\begin{LTXexample}[pos=r,width=4cm]
\begin{tabular}{ccc}
\toprule
名称  &  作用  &  效果  \\
\bottomrule
a   &   b    &  c  \\
\bottomrule[0.1em]  
\end{tabular}
\end{LTXexample}
}
\end{frame}

\begin{frame}[fragile]{表格的基础使用}
	\phantom{幻影}三线表中用 |\toprule[width]|、|\midrule{width}|、|\bottomrule{width}| 表示顶部的 rule 中部的 rule 以
	及 底部的 rule。width 为可选参数。表格的制作方法与 普通表格相同。\\
	\phantom{幻影}普通的表格内部无法自动换行,如果想换行,可以使用 |\makecell{...}| 命令在需要的位置上加上
	|\\|    \\
	\phantom{幻影}对于想要对表格进行斜线切割,可以使用 diagbox 宏包中的 |\diagbox{..}{...}..| 命令进行切割。
	例如
	\smallskip
	
{\setlength\ResultBoxSep{4mm}
	\begin{LTXexample}[pos=r,width=3.5cm]
\begin{tabular}{|c|c|c|}
\hline
\diagbox{a}{b}  &  c  & d  \\
\hline
d   &  e    & f	  \\
\hline
\end{tabular}\end{LTXexample}
}
\end{frame}

\subsection{表格的合并与复杂表格}
\begin{frame}[fragile]{表格的合并(列合并)}
	\textcolor{red}{列合并:} \\
	使用 |\multicolumn{列数}{新列格式}{内容}| 可用于将一行中几个不同的表格合并为一项。但是需要注意的是它会
	导致原来表格的列格式失效,比如原先加入竖线则需要在命令的新列格式中加入新的竖线。
	\smallskip

{
\setlength\ResultBoxSep{3mm}
\begin{LTXexample}[pos=r,justification=\centering,width=3.5cm]
\begin{tabular}{|r|r|}
\hline
\multicolumn{2}{|c|}{成绩}  \\ 
\hline
语文  &  数学  \\
\hline
87     &  100  \\
\hline
\end{tabular}
\end{LTXexample}	
}
\end{frame}

\begin{frame}[fragile]{表格的合并(列合并)}

\textcolor{red}{如果不在 } |multicolumn| \textcolor{red}{的列选项中加入竖线则会导致这种情况的出现}
\smallskip
{	
\setlength\ResultBoxSep{3mm}
	\begin{LTXexample}[pos=r,justification=\centering,width=3.5cm]
\begin{tabular}{|r|r|}
\hline
\multicolumn{2}{c}{成绩}  \\ 
\hline
语文  &  数学  \\
\hline
87     &  100  \\
\hline
\end{tabular}
\end{LTXexample}	
}
表格中,列与列之间最小距离一半是由变量 |\tabcolsep| 和 |\arraycolsep| 控制的,在处理过长表格时可以考虑在某个
环境中对这两个变量进行单独设置以使他作用于专门的表格。
\end{frame}

\begin{frame}[fragile]{表格的合并(行合并)}
行合并通常需要 |\multirow|  和 |\cline{i-j}| 配合使用,它有两种基本的用法
\begin{itemize}
\item	|\multirow{<行数>}{<宽度>}{<内容>}| 	
\item 	|\multirow{<行数>}*{<内容>}|
\end{itemize}
前者产生的表项的宽度就是输入内容的宽度,后者达到宽度后会自动换行,而表格行之间的间距由 |\arraystretch| 来
控制,可以重新定义其比例系数。
\bigskip
\begin{LTXexample}[pos=r,width=4cm]
\begin{tabular}{|c|r|r|}
\hline
\multirow{2}*{姓名}  &
\multicolumn{2}{c|}{成绩}\\   
\cline{2-3}
& 语文   &  数学  \\  
\hline
张三    &   87  &  100  \\
\hline
\end{tabular}
\end{LTXexample}
\end{frame}


\begin{frame}[fragile]{长表格与表格工具的使用}
对于长表格,需要使用 |\longtable| 宏包,定制起来较为麻烦,更建议使用 csv 相关宏包进行定制。
\textcolor{red}{下面是常用的表格处理工具与宏包}
\begin{table}[htbp]
\begin{tabular}{ll}
	\toprule[0.1em]
	工具		&		介绍		\\
	\midrule
	Excel2LaTeX  &  一款 Excel 的宏,下载方便,使用起来也比较简单	\\
	GenerateTable &	可以在线生成 \LaTeX 表格的网站,使用起来也比较方便	\\
	pgfplotstable	&	可以读入数据和内嵌数据排版		\\
	csvsimple       &    通过 csv 文件保存数据后读入制作表格 	\\
	datatool      	  &    	通过 csv 文件保存数据后读入制作表格 	\\
	\hline
\end{tabular}
\end{table}
\end{frame}

\begin{frame}[fragile]{制作表格总结}
	\begin{itemize}
		\item  在制作表格时,最好将表格放入浮动题环境内即 table 环境内,让表格自由的浮动以适应排版
		\item   table 环境内插入 |\caption| 和 |\label| ,label 必须放在 caption 之后
		\item   三线表尽量不要再加多余竖线
		\item   制作表格需要靠时间和经验的积累,一朝一夕无法学成
	\end{itemize}
\end{frame}