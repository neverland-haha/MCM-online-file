%最后的几章
\section{参考文献}
\subsection{使用 thebibliography }

\begin{frame}[fragile]{使用thebibliography 环境}
由于美赛中涉及到中文参考文献翻译的问题,可以先选择符合 gbt7714-2015 样式的引用方式之后,再利用翻译软件
进行翻译,写入 bibitem 之中。
\begin{lstlisting}[basicstyle=\tiny]
\begin{thebibliography}{99}
\bibitem{1} D.~E. KNUTH   The \TeX{}book  the American
Mathematical Society and Addison-Wesley
Publishing Company , 1984-1986.
\bibitem{2}Lamport, Leslie,  \LaTeX{}: `` A Document Preparation System '',
Addison-Wesley Publishing Company, 1986.
\bibitem{3}\url{https://www.latexstudio.net/}
\end{thebibliography}
\end{lstlisting}
|thebibliography| 旁的参数为参考文献的上限条数,最高为 99。如果需要引用则可以引用 |\cite{key}| 指令进行引用。
\end{frame}

\begin{frame}[fragile]{使用 bibtex 管理参考文献}
\begin{lstlisting}
\bibliographystyle{style}
\bibliography{bib file}
\end{lstlisting}
该方法需要创建 *.bib 文件,并且需要管理 bib 文件中的文献条目,可以使用谷歌学术镜像、必应学术、百度学术、
知网等导出 bib 条目,然后利用合适的编译链进行编译。对于美赛的参考条目,可以尝试将 author、title、journal 翻
译改为英文,但无需使用 gbt7714 宏包。但是对于美赛中有较多的网页,引用较为不便。
\end{frame}

\section{插入代码}
\begin{frame}[fragile]{插入代码}
\phantom{幻影}由于美赛与国赛不同,要求不提交支撑材料,所以书写的代码需要以 lstlisting 环境插入,而使用者只
需要将代码放入 code 文件夹中并使用 |\lstinputlisting[language=...]{./code/filename}|	
命令就可以插入相应的代码,demo 中举了 c++以及 
MATLAB 
的样式,如果需要设置其他语言,可以将方括号内的 language 改为相应的语言即可。支持的语言可以利用命令 
|texdoc listings| 查看文档的第 13 页,文档中的语言都支持。
\end{frame}

\begin{frame}{写在最后的话}
\phantom{幻影}数学建模应该是大多数国人接触 \LaTeX 的第一次机会,当你真正会用它之后,你会爱上用它,逐
渐摒弃 word。\LaTeX 原本是书写《程序的艺术》的排版语言,它也成为了一种艺术的程序,如果有机会继续使
用下去,你会逐渐发现它的好处。\\
\phantom{幻影} 回头再去看一些注意事项,你会发现文档远比百度好用,学会提问的重要性。\\
\phantom{幻影幻影幻幻影} \textbf{欢迎关注 LaTeX 工作室的公众号}
\begin{figure}
\centering
\includegraphics[scale=0.3]{gongzhonghao.jpg}
\end{figure}
\phantom{幻影幻影幻影} \textcolor{red}{\Large 最后祝大家在比赛中取得好成绩!}

\end{frame}