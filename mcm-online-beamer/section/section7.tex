%--------------------------------------------------------插图章节-----------------------------------------------------------------------
\section{\LaTeX 插图}
\subsection{插图}

\begin{frame}[fragile]{\LaTeX 插图}
关于插图,也有许多的奥秘,如果不会插图建议阅读下面这篇文档。
\begin{figure}[htbp]
\centering
\includegraphics[scale=0.2]{chatu.png}
\end{figure}
这篇文档应该能够解决你在建模时候的所有建模问题,合理利用即可。
\end{frame}
 

 \begin{frame}[fragile]{\LaTeX 插图}
\begin{lstlisting}
\begin{figure}[htbp]
\centering
\includegraphics[选项]{文件名}
\caption{...}
\label{...}
\end{figure}
\end{lstlisting}
\begin{itemize}
	\item 图片有矢量图和位图,矢量图建议以 *.pdf 格式插入,位图常用的有 *.png、*.jpg、*.jpeg等。并不建
	议插入 eps 图形。
	\item 在选项中常用的主要有 height、width、 scale、 angle 几个选项。主要控制插入图片的高度、宽度、
	缩放的尺寸以及角度。
	\item figure 与 table 是一种浮动体环境,\LaTeX 主要先排版文字再排版浮动题内的内容,figure 与 table 以及 
	algorithm 环境都都有可选参数 h、t、b、p,当四个组合时排版效果最好。如果想想限定在制定位置需要使用
	|[!htpb]|,若排版允许,则会固定位置,若页面排版不允许,则浮动体内容再次浮动
\end{itemize}

\end{frame}

% %------------------------------------------------------并排子图共享一个标题---------------------------------------------------------
\begin{frame}[fragile]{并排子图共享一个标题}
\begin{lstlisting}
\begin{figure}[htbp]
\centering
\includegraphics[width=1in]{filename}
\hspace{1in}
\includegraphics[width=1in]{filename} 
\caption{Two Graphics in One Figure}
\end{figure} \end{lstlisting}	
常见的并排图形有3种情况,此为其中一种,即两张图共享一个 caption, |\centering| 使得图形居中, 而内部的 
|\hspace{length}| 设置了两张图之间的水平间距,也可以使用 |\hfil| 代替,使得两张图推向两边
\end{frame}


\begin{frame}[fragile]{并排子图共享一个标题}
\begin{lstlisting}
\begin{figure}
centering \begin{minipage}[c]{0.5\textwidth}
\centering
\includegraphics[width=1in]{1.jpg}
\end{minipage}%
\begin{minipage}[c]{0.5\textwidth}
\centering
\includegraphics[width=2in]{2.jpeg}
\end{minipage}
\caption{Centers Aligned Vertically}
\end{figure}
\end{lstlisting}
其中  |minipage| 表示小页环境,它使得整行被分成几部分,而 |\textwidth| 表示整个页面版芯的宽度。需要注意的是
在这个代码种 $\%$ 无法删去,否则第二幅图会换行。  
\end{frame}

\begin{frame}[fragile]{并列的子图}
并列的子图可以使用 subfigure 宏包中的 |subfigure| 命令实现
\begin{lstlisting}[basicstyle=\tiny]
\begin{figure}
\subfigure[Small Box with a Long Caption]
{
\label{fig:mini:subfig:a} %% label for first subfigure
\begin{minipage}[b]{0.5\textwidth}
\centering
\includegraphics[width=1in]{filename} 
\end{minipage}
}% 	
\subfigure[Big Box]
{
\label{fig:mini:subfig:b} %% label for second subfigure
\begin{minipage}[b]{0.5\textwidth}
\centering
\includegraphics[width=1.5in]{filename} 
\end{minipage}
}
\caption{Minipages Inside Subfigures}
\label{fig:mini:subfig} %% label for entire figure
\end{figure}
\end{lstlisting}
实际上也是小页环境的功劳!
\end{frame}

\begin{frame}[fragile]{插图总结}
\phantom{幻影}实际上插图手册基本就能解决所有的问题,如果仍有问题可以在小屋的问答区或者QQ群询问。
在这里不再列举更多图形。\\
\phantom{幻影} 对于有强迫症的人来说,尽量适应而非尝试去改变。
\end{frame}