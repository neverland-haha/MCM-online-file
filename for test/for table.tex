\documentclass{ctexart}
\usepackage{multirow}
\usepackage{booktabs}
\usepackage{amsmath}
\usepackage{makecell}
\usepackage{diagbox}
\usepackage{cleveref}
\usepackage{csvsimple}

\begin{document}
	
%最基础的表格
\begin{tabular}{lcr}
	\hline
	1 		&		2 		&   3		\\
	\hline
	一	  &	 	  二		&	三		\\
	\hline
\end{tabular}	

\vspace{1em}
\begin{tabular}{|l|c|r|}
	\hline
	1 		&		2 		&   3		\\
	\hline
	一	  &	 	  二		&	三		\\
	\hline
\end{tabular}		

%设置间距
\vspace{1em}
{	\setlength\tabcolsep{3em}		% 设置水平间距,列间距的一般由 tabcolsep 和 arraycolsep
\renewcommand{\arraystretch}{2}	%设置垂直间距
\begin{tabular}{|l|c|r|}
	\hline
	1 		&		2 		&   3		\\
	\hline
	一	  &	 	  二		&	三		\\
	\hline
\end{tabular}	
}

%其他列格式以及 diagbox
\vspace{1em}
\begin{tabular}{|c|*{4}{c}|}
\hline
\diagbox{天干}{地支}   &	子	&	丑	&	寅	&	卯	\\
\hline
甲	&	1	&	&	51 &	\\
乙	&	&   2	&	&	52		\\
丙	&	13	&   &	3  &	 \\
丁	&	&	14	&	&	4	\\
\hline			
\end{tabular}
	

%表格合并
%行列合并、cline 使用
	\vspace{1em}
	\begin{tabular}{|c|r|c|}
		\hline
		\multirow{2}*{姓名}   &      \multicolumn{2}{|c|}{成绩}   	\\    	\cline{2-3}
				&	语文				& 	数学 		\\
		\hline
		张三 	&		87			&     100  \\
		\hline		
	\end{tabular}

\vspace{1em}
	\begin{tabular}{|c|c|}
		\hline
		\multirowcell{3}{各科\\成绩}	
		&78		\\	\cline{2-2}
		&82 		\\	\cline{2-2} 
	    &86			\\   
	    \hline
	\end{tabular}
	\vspace{1em}
	

%正确书写表格	
\begin{table}[htbp]
\centering
\caption{测试表}
\label{table:1}
\begin{tabular}{|c|r|c|}
	\hline
	\multirow{2}*{姓名}   &      \multicolumn{2}{|c|}{成绩}   	\\    	\cline{2-3}
	&	语文				& 	数学 		\\
	\hline
	张三 	&		87			&     100  \\
	\hline		
\end{tabular}
\end{table}	
测试一下表格的引用 \cref{table:1}

%csvsimple 宏包的使用
{
	\begin{table}[htbp]
		\centering
		\caption{测试表2}
		\label{table:2}
		\csvautobooktabular{csvtest1.csv}
	\end{table}
}


\vspace{1em}
%测试csvautobooktabular
{
\begin{table}[htbp]
\centering
	\caption{测试表3}
	\label{table:3}
 \csvautobooktabular[]{csvtest.csv}
\end{table}
}



	
\end{document}